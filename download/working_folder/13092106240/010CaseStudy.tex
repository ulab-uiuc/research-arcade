\vspace{-2mm}
\section{Case Study: Out-of-distribution Use}
\label{case-study-section}

As discussed in Section~\S\ref{sec:core-results}, the node masking evaluation in \envname targets \textit{in-distribution} settings with predefined neighborhoods. In real-world use, however, \envname must generate non-existing papers and reviews without such neighborhoods, requiring automatic construction via paper–researcher matching. This leads to \textit{out-of-distribution} cases, such as interdisciplinary research, where unrelated papers and researchers form unconventional neighborhoods without prior related works.

\xhdr{\envname can inspire interdisciplinary research} Interdisciplinary research is often challenging due to limited collaboration across fields. \envname addresses this by enabling agents with diverse expertise to read, interact, and co-create novel ideas. For example, as shown in Figure~\ref{fig:case-study}, combining NLP and astronomy papers leads to using kinematic models to analyze language evolution, while linking NLP and criminology inspires the use of LLMs to support communities affected by mass incarceration. These domain pairings are rarely explored in existing literature, demonstrating \envname’s ability to generate innovative, cross-disciplinary research directions.

\xhdr{\envname-written contents might have limited use in the real world} 
\envname exhibits failure modes when combining too many disparate domains, often producing incoherent or superficial outputs. For example, combining researchers and papers from LLM, biology, criminology, and astronomy, \envname generates a research question of ``\textit{How does coded language in political discourse influence societal biases, and how can a Bayesian hierarchical model be employed to analyze this effect while simultaneously addressing observational biases in white dwarf population studies?}'' It simply strings together terminology from different domains without presenting a clear research direction. Such vagueness might hinder the real use of the papers simulated from \envname. 



