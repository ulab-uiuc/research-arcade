\begin{filecontents}{si.aux}
\relax 
\bibstyle{sn-mathphys-num}
\providecommand\hyper@newdestlabel[2]{}
\providecommand\HyperFirstAtBeginDocument{\AtBeginDocument}
\HyperFirstAtBeginDocument{\ifx\hyper@anchor\@undefined
\global\let\oldnewlabel\newlabel
\gdef\newlabel#1#2{\newlabelxx{#1}#2}
\gdef\newlabelxx#1#2#3#4#5#6{\oldnewlabel{#1}{{#2}{#3}}}
\AtEndDocument{\ifx\hyper@anchor\@undefined
\let\newlabel\oldnewlabel
\fi}
\fi}
\global\let\hyper@last\relax 
\gdef\HyperFirstAtBeginDocument#1{#1}
\providecommand\HyField@AuxAddToFields[1]{}
\providecommand\HyField@AuxAddToCoFields[2]{}
\Newlabel{1}{1}
\Newlabel{2}{2}
\Newlabel{3}{3}
\Newlabel{4}{4}
\Newlabel{5}{5}
\Newlabel{6}{6}
\@writefile{toc}{\contentsline {section}{\numberline {1}Details of solubility prediction and solvent screening}{2}{section.1}\protected@file@percent }
\newlabel{results_solubility}{{1}{2}{Details of solubility prediction and solvent screening}{section.1}{}}
\@writefile{lof}{\contentsline {figure}{\numberline {S1}{\ignorespaces Annotated screenshot of solubility prediction results in ASKCOS. The SMILES strings of the solute and the solvent are input in the \texttt  {Solute} and \texttt  {Solvent} panels. The desired temperature (in K) for the solubility prediction is input in the \texttt  {Temperature} panel. The predicted solubility and solvation properties of solute cenobamate, defined by the SMILES string \texttt  {NC(=O)OC(Cn1ncnn1)c1ccccc1Cl}, in solvent THF, defined by the SMILES string \texttt  {C1CCOC1}, at 298 K and 323 K are displayed. The property values to display (e.g., $\log S$, Abraham parameters, etc.) can be selected from the \texttt  {Select Columns} panel. \relax }}{2}{figure.caption.1}\protected@file@percent }
\providecommand*\caption@xref[2]{\@setref\relax\@undefined{#1}}
\newlabel{fig_solubility_prediction}{{S1}{2}{Annotated screenshot of solubility prediction results in ASKCOS. The SMILES strings of the solute and the solvent are input in the \texttt {Solute} and \texttt {Solvent} panels. The desired temperature (in K) for the solubility prediction is input in the \texttt {Temperature} panel. The predicted solubility and solvation properties of solute cenobamate, defined by the SMILES string \texttt {NC(=O)OC(Cn1ncnn1)c1ccccc1Cl}, in solvent THF, defined by the SMILES string \texttt {C1CCOC1}, at 298 K and 323 K are displayed. The property values to display (e.g., $\log S$, Abraham parameters, etc.) can be selected from the \texttt {Select Columns} panel. \relax }{figure.caption.1}{}}
\citation{chung_group_2022}
\@writefile{lof}{\contentsline {figure}{\numberline {S2}{\ignorespaces Annotated screenshot of solvent screening results in ASKCOS. The SMILES string of the solute is input in the \texttt  {Solute} panel. Users can select the solvents from the two predefined sets or create custom solvent sets in the \texttt  {Solvent Set} panel. The desired temperatures (in K) at which to compute the solubility and solvation properties are input in the \texttt  {Temperatures} panel. The predicted solubility for solute cenobamate defined by the SMILES string \texttt  {NC(=O)OC(Cn1ncnn1)c1ccccc1Cl} in a custom set of solvents at 298 K and 323 K is displayed. \relax }}{4}{figure.caption.2}\protected@file@percent }
\newlabel{fig_solvent_screening}{{S2}{4}{Annotated screenshot of solvent screening results in ASKCOS. The SMILES string of the solute is input in the \texttt {Solute} panel. Users can select the solvents from the two predefined sets or create custom solvent sets in the \texttt {Solvent Set} panel. The desired temperatures (in K) at which to compute the solubility and solvation properties are input in the \texttt {Temperatures} panel. The predicted solubility for solute cenobamate defined by the SMILES string \texttt {NC(=O)OC(Cn1ncnn1)c1ccccc1Cl} in a custom set of solvents at 298 K and 323 K is displayed. \relax }{figure.caption.2}{}}
\citation{guan_regio-selectivity_2021,li_when_2024,stuyver2022quantum}
\citation{li_when_2024}
\citation{RDKit}
\@writefile{toc}{\contentsline {section}{\numberline {2}Details of QM descriptor prediction}{5}{section.2}\protected@file@percent }
\newlabel{results_qm}{{2}{5}{Details of QM descriptor prediction}{section.2}{}}
\@writefile{lof}{\contentsline {figure}{\numberline {S3}{\ignorespaces Annotated screenshot of QM descriptor prediction results in ASKCOS. The predicted QM descriptors of cenobamate, defined by the SMILES string \texttt  {NC(=O)OC(Cn1ncnn1)c1ccccc1Cl}, are displayed (top). The atom and bond descriptors can also be viewed individually with the 3D visualization page (bottom). The selected atom is highlighted in yellow, and the predicted atom properties for the selected atom is displayed on the left panel. This panel also provides predicted values for each bond connected to the selected atom.\relax }}{6}{figure.caption.3}\protected@file@percent }
\newlabel{fig_qm_descriptors}{{S3}{6}{Annotated screenshot of QM descriptor prediction results in ASKCOS. The predicted QM descriptors of cenobamate, defined by the SMILES string \texttt {NC(=O)OC(Cn1ncnn1)c1ccccc1Cl}, are displayed (top). The atom and bond descriptors can also be viewed individually with the 3D visualization page (bottom). The selected atom is highlighted in yellow, and the predicted atom properties for the selected atom is displayed on the left panel. This panel also provides predicted values for each bond connected to the selected atom.\relax }{figure.caption.3}{}}
\citation{FastAPI}
\citation{TorchServe}
\citation{ASKCOSwiki}
\@writefile{toc}{\contentsline {section}{\numberline {3}Technical details of software engineering}{7}{section.3}\protected@file@percent }
\newlabel{method_software}{{3}{7}{Technical details of software engineering}{section.3}{}}
\@writefile{toc}{\contentsline {subsection}{\numberline {3.1}Refactor into a microservice-based architecture}{7}{subsection.3.1}\protected@file@percent }
\@writefile{toc}{\contentsline {subsection}{\numberline {3.2}Containerized microservices for backend modules}{7}{subsection.3.2}\protected@file@percent }
\newlabel{microservices}{{3.2}{7}{Containerized microservices for backend modules}{subsection.3.2}{}}
\citation{Vue}
\citation{Vuetify}
\citation{Cypress}
\citation{Pinia}
\citation{VueRouter}
\@writefile{toc}{\contentsline {subsection}{\numberline {3.3}Centralized API gateway}{8}{subsection.3.3}\protected@file@percent }
\newlabel{method_api_gateway}{{3.3}{8}{Centralized API gateway}{subsection.3.3}{}}
\@writefile{toc}{\contentsline {subsection}{\numberline {3.4}Frontend}{8}{subsection.3.4}\protected@file@percent }
\citation{ASKCOSwiki}
\@writefile{toc}{\contentsline {subsection}{\numberline {3.5}Application monitoring and logging}{9}{subsection.3.5}\protected@file@percent }
\@writefile{toc}{\contentsline {section}{\numberline {4}Advanced features}{9}{section.4}\protected@file@percent }
\newlabel{advanced_features}{{4}{9}{Advanced features}{section.4}{}}
\@writefile{toc}{\contentsline {subsection}{\numberline {4.1}User-friendly and customizable deployment}{9}{subsection.4.1}\protected@file@percent }
\newlabel{method_deployment}{{4.1}{9}{User-friendly and customizable deployment}{subsection.4.1}{}}
\@writefile{toc}{\contentsline {subsection}{\numberline {4.2}Model retraining and integration}{10}{subsection.4.2}\protected@file@percent }
\newlabel{method_retraining}{{4.2}{10}{Model retraining and integration}{subsection.4.2}{}}
\@writefile{toc}{\contentsline {subsection}{\numberline {4.3}User customization}{10}{subsection.4.3}\protected@file@percent }
\newlabel{method_customization}{{4.3}{10}{User customization}{subsection.4.3}{}}
\citation{NIHNameResolver}
\@writefile{toc}{\contentsline {subsection}{\numberline {4.4}API usage}{11}{subsection.4.4}\protected@file@percent }
\newlabel{method_api_usage}{{4.4}{11}{API usage}{subsection.4.4}{}}
\citation{ASKCOSwiki}
\citation{FDAApproval}
\citation{NIHNameResolver}
\@writefile{toc}{\contentsline {section}{\numberline {5}Illustration of applying ASKCOS to FDA-approved small molecule drugs from 2019 to 2023}{12}{section.5}\protected@file@percent }
\newlabel{results_fda}{{5}{12}{Illustration of applying ASKCOS to FDA-approved small molecule drugs from 2019 to 2023}{section.5}{}}
\@writefile{toc}{\contentsline {subsection}{\numberline {5.1}Initial automated tree building with typical settings}{12}{subsection.5.1}\protected@file@percent }
\@writefile{lof}{\contentsline {figure}{\numberline {S4}{\ignorespaces Shortest retrosynthetic routes suggested by ASKCOS for FDA-approved small molecule drug components, part I. Top-1 recommendations by the condition recommender (V1) are shown in blue below each arrow. Ions are written in salt form. Abbreviations are defined in Table \ref {table:abbreviations}.\relax }}{13}{figure.caption.4}\protected@file@percent }
\newlabel{fig:fda_study_1}{{S4}{13}{Shortest retrosynthetic routes suggested by ASKCOS for FDA-approved small molecule drug components, part I. Top-1 recommendations by the condition recommender (V1) are shown in blue below each arrow. Ions are written in salt form. Abbreviations are defined in Table \ref {table:abbreviations}.\relax }{figure.caption.4}{}}
\@writefile{lof}{\contentsline {figure}{\numberline {S5}{\ignorespaces Shortest retrosynthetic routes suggested by ASKCOS for FDA-approved small molecule drug components, part II. Top-1 recommendations by the condition recommender (V1) are shown in blue below each arrow. Ions are written in salt form. Abbreviations are defined in Table \ref {table:abbreviations}.\relax }}{14}{figure.caption.5}\protected@file@percent }
\newlabel{fig:fda_study_2}{{S5}{14}{Shortest retrosynthetic routes suggested by ASKCOS for FDA-approved small molecule drug components, part II. Top-1 recommendations by the condition recommender (V1) are shown in blue below each arrow. Ions are written in salt form. Abbreviations are defined in Table \ref {table:abbreviations}.\relax }{figure.caption.5}{}}
\@writefile{lof}{\contentsline {figure}{\numberline {S6}{\ignorespaces Shortest retrosynthetic routes suggested by ASKCOS for FDA-approved small molecule drug components, part III. Top-1 recommendations by the condition recommender (V1) are shown in blue below each arrow. Abbreviations are defined in Table \ref {table:abbreviations}.\relax }}{15}{figure.caption.6}\protected@file@percent }
\newlabel{fig:fda_study_3}{{S6}{15}{Shortest retrosynthetic routes suggested by ASKCOS for FDA-approved small molecule drug components, part III. Top-1 recommendations by the condition recommender (V1) are shown in blue below each arrow. Abbreviations are defined in Table \ref {table:abbreviations}.\relax }{figure.caption.6}{}}
\@writefile{lof}{\contentsline {figure}{\numberline {S7}{\ignorespaces Shortest retrosynthetic routes suggested by ASKCOS for FDA-approved small molecule drug components, part IV. Top-1 recommendations by the condition recommender (V1) are shown in blue below each arrow. Ions are written in salt form. Abbreviations are defined in Table \ref {table:abbreviations}.\relax }}{16}{figure.caption.7}\protected@file@percent }
\newlabel{fig:fda_study_4}{{S7}{16}{Shortest retrosynthetic routes suggested by ASKCOS for FDA-approved small molecule drug components, part IV. Top-1 recommendations by the condition recommender (V1) are shown in blue below each arrow. Ions are written in salt form. Abbreviations are defined in Table \ref {table:abbreviations}.\relax }{figure.caption.7}{}}
\citation{anderson_pyrrole_1985,elliott_intramolecular_2007,koovits_conformationally_2016}
\citation{boogaard_ring_1994,ornstein_4-tetrazolylalkylpiperidine-2-carboxylic_1991,price_orally_2018}
\@writefile{lot}{\contentsline {table}{\numberline {S1}{\ignorespaces List of abbreviations used in Figures \ref {fig:fda_study_1}, \ref {fig:fda_study_2}, \ref {fig:fda_study_3}, and \ref {fig:fda_study_4}\relax }}{17}{table.caption.9}\protected@file@percent }
\newlabel{table:abbreviations}{{S1}{17}{List of abbreviations used in Figures \ref {fig:fda_study_1}, \ref {fig:fda_study_2}, \ref {fig:fda_study_3}, and \ref {fig:fda_study_4}\relax }{table.caption.9}{}}
\@writefile{toc}{\contentsline {subsection}{\numberline {5.2}Step-wise verification and further analysis \emph  {within} ASKCOS}{17}{subsection.5.2}\protected@file@percent }
\newlabel{verification_and_analysis}{{5.2}{17}{Step-wise verification and further analysis \emph {within} ASKCOS}{subsection.5.2}{}}
\citation{henry1989mitsunobu}
\@writefile{lot}{\contentsline {table}{\numberline {S2}{\ignorespaces Recommended conditions for the last step in the proposed retrosynthesis of abrocitinib. The top 3 recommendations of the V2 condition recommender using fingerprint (fp) and graph representations of molecules are shown. \relax }}{18}{table.caption.11}\protected@file@percent }
\newlabel{table:abrocitinib_conditions}{{S2}{18}{Recommended conditions for the last step in the proposed retrosynthesis of abrocitinib. The top 3 recommendations of the V2 condition recommender using fingerprint (fp) and graph representations of molecules are shown. \relax }{table.caption.11}{}}
\@writefile{lot}{\contentsline {table}{\numberline {S3}{\ignorespaces Recommended conditions for the last step in the proposed retrosynthesis of asciminib. The top 3 recommendations of the V2 condition recommender using fingerprint (fp) and graph representations of molecules are shown. \relax }}{19}{table.caption.13}\protected@file@percent }
\newlabel{table:asciminib_conditions}{{S3}{19}{Recommended conditions for the last step in the proposed retrosynthesis of asciminib. The top 3 recommendations of the V2 condition recommender using fingerprint (fp) and graph representations of molecules are shown. \relax }{table.caption.13}{}}
\@writefile{lot}{\contentsline {table}{\numberline {S4}{\ignorespaces Recommended conditions for the second step in the proposed retrosynthesis of maribavir. The top 3 recommendations of the V2 condition recommender using fingerprint (fp) and graph representations of molecules are shown.\relax }}{20}{table.caption.15}\protected@file@percent }
\newlabel{table:maribavir_conditions}{{S4}{20}{Recommended conditions for the second step in the proposed retrosynthesis of maribavir. The top 3 recommendations of the V2 condition recommender using fingerprint (fp) and graph representations of molecules are shown.\relax }{table.caption.15}{}}
\@writefile{toc}{\contentsline {subsection}{\numberline {5.3}Re-running the automated Tree Builder with different search settings}{21}{subsection.5.3}\protected@file@percent }
\@writefile{lof}{\contentsline {figure}{\numberline {S8}{\ignorespaces Shortest retrosynthetic routes suggested by ASKCOS for lotilaner and pemigatinib when re-running with smaller numbers of templates per expansion step. Top-1 recommendations by the condition recommender (V1) are shown in blue below each arrow. Ions are written in salt form. Abbreviations are defined in Table \ref {table:abbreviations}.\relax }}{22}{figure.caption.16}\protected@file@percent }
\newlabel{fig:fda_study_interesting_1}{{S8}{22}{Shortest retrosynthetic routes suggested by ASKCOS for lotilaner and pemigatinib when re-running with smaller numbers of templates per expansion step. Top-1 recommendations by the condition recommender (V1) are shown in blue below each arrow. Ions are written in salt form. Abbreviations are defined in Table \ref {table:abbreviations}.\relax }{figure.caption.16}{}}
\citation{thakkar_ring_2020}
\citation{Pistachio}
\@writefile{lof}{\contentsline {figure}{\numberline {S9}{\ignorespaces Shortest retrosynthetic route suggested by ASKCOS for etrasimod when re-running with the addition of the ring breaker template set. Top-1 recommendations by the condition recommender (V1) are shown in blue below each arrow. Ions are written in salt form. Abbreviations are defined in Table \ref {table:abbreviations}.\relax }}{23}{figure.caption.17}\protected@file@percent }
\newlabel{fig:fda_study_interesting_2}{{S9}{23}{Shortest retrosynthetic route suggested by ASKCOS for etrasimod when re-running with the addition of the ring breaker template set. Top-1 recommendations by the condition recommender (V1) are shown in blue below each arrow. Ions are written in salt form. Abbreviations are defined in Table \ref {table:abbreviations}.\relax }{figure.caption.17}{}}
\@writefile{lof}{\contentsline {figure}{\numberline {S10}{\ignorespaces Shortest retrosynthetic route suggested by ASKCOS for pralsetinib when re-running with the addition of a template-free strategy. Top-1 recommendations by the condition recommender (V1) are shown in blue below each arrow. Ions are written in salt form. \textsuperscript  {1}The original shortest route contains an alchemical step here (not shown) proposed by the Transformer model using a reactant with an extra nitrogen in the ring that has no way of disappearing in the product; excluding this reactant with one-click filters out all routes with this nonsensical step, and the replacement step from the new shortest route after filtering is shown. \textsuperscript  {2}These two steps are obtained from manual expansion in the IPP upon rejecting the Transformer-proposed step marked with a crossed out arrow. Abbreviations are defined in Table \ref {table:abbreviations}.\relax }}{24}{figure.caption.18}\protected@file@percent }
\newlabel{fig:fda_study_interesting_3}{{S10}{24}{Shortest retrosynthetic route suggested by ASKCOS for pralsetinib when re-running with the addition of a template-free strategy. Top-1 recommendations by the condition recommender (V1) are shown in blue below each arrow. Ions are written in salt form. \textsuperscript {1}The original shortest route contains an alchemical step here (not shown) proposed by the Transformer model using a reactant with an extra nitrogen in the ring that has no way of disappearing in the product; excluding this reactant with one-click filters out all routes with this nonsensical step, and the replacement step from the new shortest route after filtering is shown. \textsuperscript {2}These two steps are obtained from manual expansion in the IPP upon rejecting the Transformer-proposed step marked with a crossed out arrow. Abbreviations are defined in Table \ref {table:abbreviations}.\relax }{figure.caption.18}{}}
\citation{CASContent}
\citation{antonchick_direct_2013}
\@writefile{toc}{\contentsline {subsection}{\numberline {5.4}Contextualizing the limitation of automatic planning}{25}{subsection.5.4}\protected@file@percent }
\@writefile{lof}{\contentsline {figure}{\numberline {S11}{\ignorespaces Sample targets for which ASKCOS fails to propose any routes with buyable building blocks within the defined search criteria from all four automatic runs.\relax }}{26}{figure.caption.19}\protected@file@percent }
\newlabel{fig:fda_study_fail}{{S11}{26}{Sample targets for which ASKCOS fails to propose any routes with buyable building blocks within the defined search criteria from all four automatic runs.\relax }{figure.caption.19}{}}
\bibdata{sn-bibliography}
\bibcite{chung_group_2022}{{1}{2022}{{Chung et~al.}}{{}}}
\bibcite{guan_regio-selectivity_2021}{{2}{2021}{{Guan et~al.}}{{}}}
\bibcite{li_when_2024}{{3}{2024}{{Li et~al.}}{{}}}
\bibcite{stuyver2022quantum}{{4}{2022}{{Stuyver and Coley}}{{}}}
\bibcite{RDKit}{{5}{}{{}}{{}}}
\bibcite{FastAPI}{{6}{}{{}}{{}}}
\bibcite{TorchServe}{{7}{}{{}}{{}}}
\bibcite{ASKCOSwiki}{{8}{}{{}}{{}}}
\bibcite{Vue}{{9}{}{{}}{{}}}
\bibcite{Vuetify}{{10}{}{{}}{{}}}
\bibcite{Cypress}{{11}{}{{}}{{}}}
\bibcite{Pinia}{{12}{}{{}}{{}}}
\bibcite{VueRouter}{{13}{}{{}}{{}}}
\bibcite{NIHNameResolver}{{14}{}{{}}{{}}}
\bibcite{FDAApproval}{{15}{}{{}}{{}}}
\bibcite{anderson_pyrrole_1985}{{16}{1985}{{Anderson et~al.}}{{}}}
\bibcite{elliott_intramolecular_2007}{{17}{2007}{{Elliott et~al.}}{{}}}
\bibcite{koovits_conformationally_2016}{{18}{2016}{{Koovits et~al.}}{{}}}
\bibcite{boogaard_ring_1994}{{19}{1994}{{Boogaard et~al.}}{{}}}
\bibcite{ornstein_4-tetrazolylalkylpiperidine-2-carboxylic_1991}{{20}{1991}{{Ornstein et~al.}}{{}}}
\bibcite{price_orally_2018}{{21}{2018}{{Price et~al.}}{{}}}
\bibcite{henry1989mitsunobu}{{22}{1989}{{Henry et~al.}}{{}}}
\bibcite{thakkar_ring_2020}{{23}{2020}{{Thakkar et~al.}}{{}}}
\bibcite{Pistachio}{{24}{}{{}}{{}}}
\bibcite{CASContent}{{25}{}{{}}{{}}}
\bibcite{antonchick_direct_2013}{{26}{2013}{{Antonchick and Burgmann}}{{}}}
\gdef \@abspage@last{29}
\end{filecontents}
